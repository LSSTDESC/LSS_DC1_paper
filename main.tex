%
% ======================================================================
\RequirePackage{docswitch}
% \flag is set by the user, through the makefile:
%    make note
%    make apj
% etc.
\setjournal{\flag}

%\documentclass[\docopts]{\docclass}

% You could also define the document class directly
\documentclass[twocolumn]{aastex62}
% Custom commands from LSST DESC, see texmf/styles/lsstdesc_macros.sty
\usepackage{lsstdesc_macros}
\usepackage{tikz}
\usetikzlibrary{calc}
\usepackage{mwe}
\usepackage{graphicx}
\graphicspath{{./}{./figures/}}
\bibliographystyle{aj}
\extrafloats{100}
\usepackage{hyperref}
% Add your own macros here:
%\newcommand{\arcmin}{\unit{arcmin}}
%\newcommand{\arcsec}{\unit{arcsec}}
\newcommand{\rachel}[1]{{\textcolor{cyan}{{\textbf (RM: #1)}}}}


%
% ======================================================================

\begin{document}

\title[LSST DESC DC1]{The LSST DESC Data Challenge 1: Generation and Analysis of Synthetic Images for Next Generation Surveys }

%\maketitlepre

\begin{abstract}

The success of the Large Synoptic Survey Telescope (LSST) as a dark energy experiment will depend on controlling systematic biases in cosmological probes. Simulations are critical for developing the methodology to estimate and mitigate these systematics. In the first Data Challenge from the LSST Dark Energy Science Collaboration, we evaluate potential systematic biases in observables, with an emphasis on galaxy clustering. We simulate LSST images, then process and analyze them using the current version of the LSST Data Management pipeline. Then we characterize the resulting systematics and implement corrections. We also show that dithering reduces the impact of potential systematic effects. Our results demonstrate that we can generate realistic LSST-like simulated images and control the systematic effects, after processing these images, at a sufficient level to enable major advances in our knowledge of dark energy and cosmology. The methodology presented here can be easily translated to current and future imaging surveys.
\end{abstract}

% Keywords are ignored in the LSST DESC Note style:
\dockeys{large-scale structure of the universe}

%\maketitlepost

%\maketitle
% ----------------------------------------------------------------------
%

\section{Introduction}
\label{sec:intro}
\rachel{Should enlarge font in tick and axis labels to be comparable to font size in the paper
  text in all of the following figures: 4, 5, 6, 7, 8, 9, 11, 12, 13, 18, 19, 20.}
\rachel{For plots with many colored lines, I strongly recommend using a color blind-friendly color palette.  See section 17.1 of the DESC style guide https://github.com/LSSTDESC/Style\_Guide/raw/compiled/Style\_Guide.pdf, which has some clear and easy-to-adopt recommendations complete with code snippets, so it should be an easy change.}

The increase in statistical power from recent cosmological experiments makes the modeling and mitigation of systematic uncertainties key to extracting the maximum amount of information from these surveys. More traditional in high energy particle physics~\citep{Brun:118715, 2006JHEP...05..026S}, end-to-end simulations provide a unique framework to
model systematics and streamline processing and analysis pipelines given out complete understanding of the inputs and outputs. With the increasing availability of computational resources, this approach has also been extended to photometric redshift galaxy surveys~\citep{2016MNRAS.457..786S,2016ApJ...817...25B} \rachel{We should rethink how we frame this.  First, DES is not just a photometric redshift survey -- let's call it an imaging survey.  Second, Balrog is not actually an end-to-end {\em ab initio} simulation tool as is implied here.  So we need to either make the text a little more nuanced and refer to the concept of injection simulations, or remove the Suchyta ref.}, and similar efforts are seen in spectroscopic surveys such as DESI~\citep{2016arXiv161100036D}.

For surveys like the LSST~\citep{Overview}, where the expected data volume is very large, and where a highly stringent control of the systematic uncertainties is required, producing these
kind of end-to-end simulations enables successful validation and verification of the processing and
analysis pipelines. With $\sim 50$ PB of raw data and $\sim 40$ billion objects~\citep{Overview} the
data handling by itself becomes challenging. \rachel{Those numbers are a bit deceptive, because they
  are for the entire LSST survey, right?  Whereas I think the point we should emphasize that
  even the first year of LSST data, or simulations of a small subset of the LSST area, as we are doing here, comes with challenges.} There are many approaches to generate these end-to-end simulations. Usually, one would start from a source catalog where the objects are modeled according to certain parameters (sizes, shapes, fluxes), then render images from these sources, add some observational effects and, finally, use a software of choice to detect and measure the different properties of these images. Depending on the approach used to generate these images, some difficulties may arise when trying to relate inputs and outputs, especially in a very deep survey like LSST.

The LSST Dark Energy Science Collaboration (DESC\footnote{\url{http://lsstdesc.org/}}) has planned a
series of Data Challenges (DCs) carried out over a period of years, aimed at successively
more stringent and comprehensive tests of analysis pipelines, to ensure adequate control of
systematic uncertainties for analysis of the LSST data.  An additional goal of these DCs is
development of the infrastructure for analyzing, storing, and serving substantial data volumes; even
while using the outputs of the LSST Data Management (DM) stack as inputs into analysis pipelines,
the analysis pipelines will need to handle quantities of data beyond those seen by ongoing surveys,
even after just a single year of LSST data is available.  Moreover, it is anticipated that
non-negligible subsets of the data may need to be reprocessed to generate systematic error budgets
(e.g., assessing sensitivity of the results to certain stages of the analysis process by changing
some parameters in the analysis).  While these DCs may in principle use real
data from precursor datasets, at least a portion of each one will involve the production of
simulated LSST data.  Both goals of the DCs dictate a gradual increase in the sophistication and
volume of the simulated data from the first of these data challenges (DC1) to DC3.

In this paper, we present and analyze simulated images from DC1; these resemble the data that will be produced by
LSST~\citep{Overview} after 10 years of operation in $r$-band using state of the art tools produced by DESC and LSST. \rachel{Mention area difference?} We characterize the products of this process, measuring photometric, astrometric and clustering properties of the sample. These products encompass single-visit and coadded calibrated exposures (i.e., flattened, background subtracted, etc.) and source catalogs that take up to $\sim 225$ TB of disk space. We perform the 2-point clustering analysis in real and harmonic space for these simulations and assess the impact of potential systematics.

\textcolor{red}{MORE INTRODUCTION!!!} \rachel{Unclear how much more is necessary given the additions
above.}

This paper is structured as follows: \secref{design} includes a summary of the factors that informed
the design of this data challenge.  In \secref{inputs}, we describe the inputs for our simulated images. In \secref{dithering}, we discuss the dither strategies used for this study. In \secref{image_generation_pipeline}, we describe the process used to generate LSST-like artificial images. In \secref{catalogs}, we describe the processed data products generated and perform several validation tests. In \secref{analysis}, we present the clustering analyses on the simulated data products. Finally, in \secref{conclusions}, we present some concluding remarks.

\section{Data Challenge Design}
\label{sec:design}

As mentioned in \secref{intro}, the design of this first data challenge (DC1) is driven by a
combination of several factors: the need to test analysis pipelines for control of systematic
uncertainty, and to develop infrastructure for processing and serving data in a way that is useful
to DESC.  While DC2 and DC3 are intended to be full-collaboration activities, DC1 is more limited in
scope and focus, with a goal of enabling tests of the large-scale structure (LSS) working group
pipelines for analysis and systematics mitigation of two-point clustering statistics.

\begin{figure}
\centering
\includegraphics[width=0.9\columnwidth]{footprint.png}
\caption{Footprint of the DC1 dataset. We simulate 4 LSST full focal plane pointings which roughly corresponds to 40 deg$^{2}$.}
\label{fig:footprint}
\end{figure}

As a stepping stone to eventually producing images covering hundreds to thousands of square degreees
in DC2 and DC3, DC1 covers a $40$ deg$^2$ footprint.  This is sufficient to enable tests of
two-point clustering statistics up to $\sim 1$ degree scales.  To ensure the simulated image volume is
tractable, DC1 only includes images in a single band ($r$-band), but goes to full LSST 10-year
depth. The final footprint can be seen in \figref{footprint}.  We simulate observations within this
footprint using the
\texttt{minion\_1016}\footnote{\url{https://www.lsst.org/scientists/simulations/opsim/opsim-v335-benchmark-surveys}}
simulated observing cadence generated with the LSST Operations Simulator
\citep[OpSim;][]{2014SPIE.9150E..15D}.  While translational and rotational dithering will be employed,
we enable tests of the impact of dithering by generating two sets of images, both without and with
the dithering strategy described in more detail in \secref{dithering}.

\rachel{It would be valuable to have a list of steps here in the DC1 simulation generation and
  processing, explaining how these map in more detail onto paper sections.  A flowchart might be
  even more useful.}


%One of the most critical aspects in Stage IV experiments is the characterization of their instrumentation and systematic effects~\citep{2006astro.ph..9591A}. In the case of  LSST~\citep{2008arXiv0805.2366I,ScienceBook,WhitePaper} this becomes especially difficult given its wide variety of cosmological probes. In this paper we present a methodology to characterize the LSST large-scale structure (LSS) transfer function and an analysis of potential systematic effects present in LSS analyses. \CHECK{rewrite}
% ---------------------------------------------------------------------
\section{Image generation: input catalog}
\label{sec:inputs}
Image simulations allow us to study in detail the detection and deblending properties of a given image-processing pipeline. For example, if we produce images using an object catalog with random positions uniformly distributed across the sky, as well as uniformly random shapes and fluxes, we can get information about detection efficiencies as a function of flux.  However, the information about blending will not be realistic and we will not be able to capture some correlations present in real data. On the other hand, using N-body simulations as the input to generate artificial images allows us to study all the aforementioned effects. This is why we used the \texttt{CatSim}~\citep{2010SPIE.7738E..1OC,2014SPIE.9150E..14C} catalog as our input.  \texttt{CatSim} is a set of simulations provided by the LSST Simulations Team representing a realistic distribution of both Milky Way and extra-galactic sources.In particular, the extra-galactic catalog contains galaxies covering the redshift range $0 < z < 6$ in a 4.5$\times$4.5 degree footprint. The galaxies are generated by populating the dark matter haloes from the Millennium simulation~\citep{2005Nature.435.629S} using a semi-analytic baryon model described in \citet{2006MNRAS.366..499D} including magnitudes BVRIK and bulge-to-disk ratios. For all sources, a spectral energy distribution (SED), is fit to the galaxy colors using \citet{2003MNRAS.344.1000B} spectral synthesis models. Fits are undertaken independently for the bulge and disk and include inclination dependent reddening. Morphologies are modeled using two S\'{e}rsic profiles~\citep{1963BAAA....6...41S} and a single point source (for the AGN). Half-light radii for the bulge components are derived from the absolute-magnitude vs half-light radius relation given by \citet{2011A&A...534A...3G}. Stars are represented as point sources and are drawn from the Galfast model~\citep{2008ApJ...673..864J}. More information about these catalogs can be found at the LSST Simulations webpage\footnote{\url{https://www.lsst.org/scientists/simulations/catsim}}

For the Data Challenge 1 (DC1), the extra-galactic catalog was tiled to generate a $\sim 40$ deg$^{2}$ footprint. This approach introduces a periodicity that induces extra correlations in our sample, however, we confirmed that these effects appear at scales larger than we are able to map given the DC1 area.

After tiling, the input catalog contains approximately $63.1$ million sources of which, $61.3$ million are galaxies whose redshift and magnitude distributions are depicted in \figref{catalog_plots} and $1.8$ million are stars. We simulate images in $r$-band to LSST full depth ($10$ years). The final footprint can be seen in \figref{footprint}. We simulate observations within this footprint using the \texttt{minion\_1016}\footnote{\url{https://www.lsst.org/scientists/simulations/opsim/opsim-v335-benchmark-surveys}} simulated observing cadence generated with the LSST Operations Simulator (OpSim)~\citep{2014SPIE.9150E..15D}.

\begin{figure}
\centering
\includegraphics[width=0.9\columnwidth]{N_z_DC1.pdf}
\includegraphics[width=0.9\columnwidth]{N_m_DC1.pdf}
\caption{Redshift (top) and magnitude (bottom) distribution for the galaxies used as inputs for the Data Challenge 1 simulations. In the magnitude distribution we include, as references, the typical depth for a single exposure (red dashed line) and the median depth in the DC1 dithered simulation (black dashed-dotted line).}
\label{fig:catalog_plots}
\end{figure}

\section{Dither strategy}
\label{sec:dithering}

As mentioned in \secref{design}, we use OpSim's output, which contains a realization of the LSST observing cadence and the survey footprint. Since OpSim divides the sky with hexagonal tiles, the nominal telescope pointings lead to overlapping regions across adjacent tiles that are observed more often than the non-overlapping part of the field-of-view (FOV), resulting in depth non-uniformity on the scale of $\sim$1 degree and consequent systematic uncertainties \citep{2016ApJ...829...50A}. In an effort to mitigate these effects, we implement \textit{dithers} -- offsets in the nominal telescope pointings. Specifically, here we use \textit{large}, i.e., as large as the FOV, random translational dithers, implemented after every visit, and random rotational dithers implemented after every filter change. The specific translational dither strategy is chosen based on a more extensive study of the various (translational) dither strategies in \citet{2016ApJ...829...50A}, where random dithers after every visit are found to be amongst the most effective.

For our purposes, we consider both the undithered and the dithered observing strategy. For the dithered strategy, some visits will contain sensors that fall out of the DC1 region; these sensors were not simulated in order to save computational resources.

\section{Image generation and processing}
\label{sec:image_generation_pipeline}
% ---------------------------------------------------------------------

The artificial generation of astronomical images is a complex and computationally demanding process. In the recent
years, there have been major efforts in the community to create software that enables the generation of astronomical images, with various choices made in terms of level of complexity and fidelity, and computational efficiency, such as \texttt{BALROG}~\citep{2016MNRAS.457..786S}, \texttt{UFIG}~\citep{2016ApJ...817...25B}, and \textsc{PhoSim}~\citep{2015ApJS..218...14P}. In our case, we model the input sources using imSim\footnote{\url{https://github.com/LSSTDESC/imSim}}\textcolor{red}{Add reference Walter et al., in prep??}, which uses \textsc{GalSim}~\citep{2015A&C....10..121R} as a library for image rendering, and uses LSST-specific information (e.g., about the geometry of the CCDs and the focal plane, the system throughputs in the different bands, etc.) to generate LSST-like images.

\subsection{imSim}
\label{sec:imsim_pipeline}

imSim is an open-source image simulation software package that uses GalSim with a modular approach. It allows the user to change the level of realism of the simulations by changing the complexity of PSF, background and source modeling. In particular, we used a pre-release version specifically meant to perform the DC1 simulations: \texttt{imSim v.0.1.0}\footnote{\url{https://github.com/LSSTDESC/imSim/releases/tag/0.1.0}}.

For DC1, we simulate each CCD of the focal plane individually, and instead of the planned two 15-seconds exposures~\citep{Overview}, we generate a single image with a 30-second exposure time to simplify the data handling. We omit instrumental effects and variability in the optical model across the focal plane. Since this is the first DESC data challenge, we want to ensure that we are able to generate and process the simplest cases, and then, build upon this base, increasing the complexity and level of realism for future data challenges. Our sky brightness model is based off the \citet{1991PASP..103.1033K} model provided by OpSim, refined by the detailed wavelength dependence of the phenomenological model from~\citet{2016SPIE.9910E..1AY}. The PSF model is a Gaussian for the system with a full-width half-maximum airmass dependence\footnote{From LSST-20160 eqn.~(4.1) \rachel{should link to this document / name it}}, this is done to mimic the degradation in the image quality due to, e.g., gravity load\footnote{See LSE-30~\url{http://ls.st/lse-30} p.80}. A Kolmogorov profile is used to model the atmosphere which is also airmass dependent\footnote{From LSST-20160 eqn.~(4.2)}. The airmass, $X$, depends on the angular distance to the zenith, $Z$, as follows~\citep{1991PASP..103.1033K}:
\begin{equation}
X = (1 - 0.96\sin{Z})^{-0.5}.
\end{equation}
imSim can generate three different types of objects: stars, which are modeled as PSF-like objects; galaxies, which are modeled as composite (bulge plus disk) S\'{e}rsic profiles~\citep{1963BAAA....6...41S} using
the parameters given by CatSim; and AGNs which are also modeled as point sources and, for simplicity, without any variability. Future versions of imSim will have the ability to generate more complex galaxy morphologies. The brightness for these sources is computed using the magnitudes from CatSim, which are converted to counts using the latest version of the LSST throughputs\footnote{\url{https://github.com/lsst/throughputs}}. We clip the objects at magnitude 10 in order to improve the computational efficiency.

The final products of this pipeline are FITS images with information about the observing conditions. We generated more than 200,000 images in total (including both the \textit{dithered}, and \textit{undithered} fields). The average time to simulate each CCD is $\sim 4300$ seconds and the total production time is $\sim 270,000$ CPU-hours.

% ----------------------------------------------------------------------

%\subsection{PhoSim}
%\label{sec:phosim_pipeline}

%PhoSim is a complementary approach where we use the photon-shooting software \textsc{PhoSim} to create simulated images.
%\textcolor{red}{Describe PhoSim. PhoSim inputs and differences with imSim. What are we adding?}
% ----------------------------------------------------------------------
\subsection{Image processing}
\label{sec:image_processing_pipeline}

The outputs of these simulations are then processed using the LSST data management (DM) stack~\citep{Overview,ScienceBook,WhitePaper,2018PASJ...70S...5B,2015arXiv151207914J} using version 13.0\footnote{\url{https://pipelines.lsst.io/releases/v13_0.html}}. The DM stack is an open-source, high-performance data processing and analysis system intended for use in optical and infrared survey data. The code can be found at \url{dm.lsst.org} and \url{pipelines.lsst.io}. The raw, uncalibrated single exposures are used as inputs. The software performs the reduction, detection, deblending and measurement on individual visits. It then combines the single-visit images to produce the so-called coadds. The DM stack provides calibrated images and source catalogs for the individual visits and coadds stored in \texttt{FITS} files. In total, we detect and measure $\sim 10.6$ (9.7 for the undithered simulation) million objects with position, flux and shape information. We activated optional extensions for the pipeline to include \texttt{CMODEL} fluxes (see \cite{2018PASJ...70S...5B} for more details) and \texttt{HSM} shapes~\citep{2003MNRAS.343..459H,2005MNRAS.361.1287M}.
\textcolor{red}{Ping Jim Chiang/Chris Walter to write something about the processing workflows for DC1.}
An example coadd cutout is shown in \figref{coadd_example}.

\rachel{Big picture comment: we don't talk at all about any of the software that drove these things - workflows for the DM processing etc.  Is that sufficiently trivial that it doesn't need to be mentioned?}

\begin{figure}
\centering
\includegraphics[width=0.9\columnwidth]{sample_coadd_DC1.pdf}
\caption{Example of a 500 $\times$ 500 pixel cutout from a full depth coadd showing the large density of objects in the LSST full-depth images.}
\label{fig:coadd_example}
\end{figure}

\section{Output catalogs and validation}
\label{sec:catalogs}
After being processed, the catalogs are stored at the National Energy Research Scientific Computing Center (NERSC)\footnote{\url{http://www.nersc.gov}} and accessible to DESC collaborators. We generate \texttt{pandas
dataframes}~\citep{mckinney-proc-scipy-2010} and databases for each one of the coadds and input catalogs, providing collaborators some flexibility in data access for their own analyses. As mentioned earlier, the output catalogs contain 10.6 (9,7) million objects covering an area
of $\sim$43 deg$^{2}$. The catalogs include information about position, size, shape and magnitude for every object. They include several flags that give information about the presence of interpolated/saturated pixels in the objects and whether or not these objects are close to the edge of a CCD.

In order to check the level of realism and the accuracy of the processed catalogs we perform several quality assurance tests. These tests will be performed in the dithered simulation, however, unless stated, the procedures and results are totally equivalent for the undithered simulation.

\rachel{Big picture comment: you said earlier that both dithered and undithered simulations are generated and processed, but in the paragraph above and many other places you just quote a single number of objecst in the catalogs.  So is it that most places refer to dithered, and undithered ones are only used in specific subsections?  If that's true, this should be explained somewhere prominent.}

%\subsection{Astrometry checks}
%\label{sec:astrometry_checks}

%Biases in astrometry can potentially affect both clustering and weak lensing measurements. We follow two approaches to check the quality of the astrometric solutions that we obtained: an \textit{external} check comparing to the input \textit{truth} catalog; and an \textit{internal} check comparing with different visits.

%\subsubsection{External checks}
%\label{sec:external_astrometry}

%As we have already mentioned, one of the advantages of using simulations is that we have access to the \textit{true} underlying information. We use this information to check the precision of the astrometric measurements in single exposures and coadds. We select stellar objects using the classifier included in the LSST software stack\footnote{To see more details about the classifier, refer to section 4.9.10 in~\citet{2017arXiv170506766B}} and choose objects with \texttt{base\_ClassificationExtendedness\_value==0}.
%We also require that \texttt{deblend\_nChild==0} to ensure that the objects are completely deblended, i.e., they are primary sources. \rachel{I thought that ``primary'' comes with other conditions, like at the edges of patches some reconciliation is done to figure out which detection is primary?} We match these objects to the stellar sources in the input catalog. In both cases we use a \texttt{KDTree}~\citep{scikit-learn} to retrieve those objects \rachel{for this matching did you use only stars rather than all objects?} in the input catalog with centroids that are within a 0.2\arcsec\ (one pixel) radius of those detected in the output catalog. We select the matched object \rachel{star?} that is closest in magnitude, only considering sources that have a magnitude difference smaller than 0.02 magnitudes. \rachel{That is a very very tight tolerance!  The magnitude errors are larger than that for many objects, so I would expect we miss many objects with this tolerance.} We discuss matching strategies in more detail in \secref{matching}. \rachel{For astrometry tests, have you considered limiting to a bright subsample of stars?  Right now the plots/calculations have a large contributions from noise due to faint objects, I would assume, and it might actually be easier to see the real structure/offsets if you only used bright stars for this plot?}

%We select a representative single visit (visit number $270675$ for the imSim dithered run) and calculate the difference between the measured and the input positions. These are represented in \figref{astrometry_a}. We see that both RA and Dec distributions are compatible with each other, indicating that there are no anisotropies in the detection. \rachel{Why is this $\Delta$RA and not $\Delta$RA~$\cos{(\text{Dec})}$?  I don't think it really makes sense to compare $\Delta$RA and $\Delta$Dec in general.}
%However, we find that the distributions are asymmetric and that the median is not zero. This effect is even more noticeable in \figref{astrometry_b}, where we accumulated the results for 50 randomly selected visits of the imSim dithered simulation. \rachel{Is there any reason to show a single visit and then 50 randomly selected ones, with the effects of interest are present at similar levels in both?  Why not just jump to the 50 randomly selected ones, and mention that if you pick a single visit the effect is still evident?} This effect is also present in the undithered imSim run. This bias arises from the fact that the objects in the images include proper motion, which is unaccounted for in our \textit{truth tables}. However, this bias is below 15 mas, which is a much smaller scale than the resolution of the input N-body simulation. \rachel{I don't think the input N-body simulation resolution is relevant; it should be sufficient to say that we don't care about galaxy clustering statistics on tiny scales and so astrometric biases on these scales are not an issue.} As a consequence we do not expect that the two point clustering statistics will be affected by this bias. We also checked the mean astrometric residual as a function of magnitude in \figref{astrometry_b}. We notice the same bias for the brightest objects (which are dominated by nearby sources), and so are the most affected by the effects of proper motion.

%\begin{figure}
%  \centering
 % \includegraphics[width=0.45\textwidth]{astrometry_single_visit_imsim_dithered_hist}
  %\includegraphics[width=0.45\textwidth]{astrometry_single_visit_imsim_dithered_hist2d}
 % \caption{Top: Distribution of the difference $\Delta=X_{\rm measured}-X_{\rm input}$ in RA (blue) and Dec
 %   (green) coordinates for a randomly selected visit ($270675$ for the dithered run). We cannot see
  %  any differences between these. The histograms are normalized such that the total sum of the
   % counts is equal to one. We measure a non-zero median, $\Delta_{\rm median} \approx -2$ mas. Bottom:
   % 2D histogram showing the bivariate distribution of the difference in RA (horizontal axis) and
   % Dec (vertical axis). The effect is similar for the undithered simulation. \rachel{Perhaps indicate the median values on the top and bottom, with lines of some sort?  Same comment applies to next figure.}}
  %\label{fig:astrometry_a}
%\end{figure}

%\begin{figure}
  %\centering
  %\includegraphics[width=0.45\textwidth]{astrometry_imsim_dithered_50visits}
  %\includegraphics[width=0.45\textwidth]{astrometry_imsim_dithered_50visits_hist2d}
  %\includegraphics[width=0.45\textwidth]{astrometry_vs_mag_imsim_50_visits}
  %\caption{Top: Distribution of the difference $\Delta=X_{\rm measured}-X_{\rm input}$ in RA (blue) and Dec (green) coordinates as in
  %\figref{astrometry_a} but, accumulating the results for 50 randomly selected visits from the dithered run. Middle: 2D histogram
  %showing the bivariate distribution of the difference in RA (horizontal axis) and Dec (vertical axis). Bottom: Mean astrometric residual
  %as a function of magnitude for RA (blue) and Dec (green). These distributions are similar for the undithered run.}
  %\label{fig:astrometry_b}
%\end{figure}

%In addition, we wanted to check if there is a preferred orientation for the differences between the input and output position in a single visit. Using the same visit as before, we show the astrometric residuals in \figref{astrometry_c}. We see that the astrometric residuals do not show any noticeable structure and appear to be mostly random with the largest contributions close to the CCD edges.

%\begin{figure}
  %\centering
  %\includegraphics[width=0.45\textwidth]{astrometry_imsim_dithered_interp}
 % \caption{Astrometic residuals measured in visit $270675$ from the imSim dithered run. The light blue squares represent the CCD chips in
 % the LSST focal plane. The base of the arrow is on the input matched object. The arrows have been augmented by a factor 3600 for visualization purposes. \rachel{I found this last sentence confusing.  Is it necessary, given that you have an arrow on the plot indicating how large is 200 mas?  I think  you could simply say ``The arrow sizes were set to make the largest ones visually evident; see upper right part of the figure for a guide to the interpretation of arrow sizes.''}}
  %\label{fig:astrometry_c}
%\end{figure}


%\subsubsection{Internal checks}
%\label{sec:internal_astrometry}

%Another important test is to ensure the internal consistency of the astrometric solutions between different exposures. We select
%a small region of the coadded area, which we will refer to as a \textit{patch} \rachel{This isn't just us - this is LSST pipeline jargon, right?}, and compare the positions of the objects detected in the coadd
%image with the positions of objects detected in individual exposures that overlap with that patch.

%In particular, we randomly choose 10 individual exposures and look for objects that fulfill the following criteria:
%\begin{itemize}
 % \item \texttt{deblend\_nChild==0}, the object has been completely deblended (it is a primary match).
%  \item \texttt{base\_PixelFlags\_flag\_edge==0}, the object is not close to an edge.
 % \item \texttt{base\_PixelFlags\_flag\_interpolatedCenter==0}, the object does not have any interpolated pixels in its center.
%\end{itemize}

%Note that, in this case, we are not requiring the objects to be classified as stars, we are omitting the cut in
%\texttt{base\_ClassificationExtendedness\_value}, but we are adding some cuts to ensure that the objects were properly measured. Once
%we perform our selection, the next step is to match the objects in the different exposures. To do so, we use the matching algorithm
%included in the LSST software stack. We calculate the mean of the difference between the position of each source
%in the coadd, $X_{{\rm coadd},i}$ and the position of the matched object in each of the exposures where it has been detected, $X_{{\rm visit_{j}},i}$
%for $j \in [1,10]$, i.e.,

%\begin{equation}
 % \Delta = \langle X_{{\rm coadd},i} - X_{{\rm visit_{j}},i} \rangle
%\end{equation}

%We only consider matched objects detected in at least 5 exposures. The resulting distribution is shown in \figref{astrometry_internal}. We see that this distribution is noticeably narrower than those shown in \figref{astrometry_a} and \figref{astrometry_b} and no apparent bias is found, indicating that the processing is consistent among different epochs. \rachel{Given that the object selection differs, I'm not sure we can ascribe any significance to the distribution being narrower, right?  Any noise contribution to the centroid determination should differ, so I don't think they can really be compared in this way.}

%\begin{figure}
 % \centering
  %\includegraphics[width=0.45\textwidth]{astrometry_internal_10visits_imsim_undithered}
  %\caption{Distribution of the mean difference in position (RA:blue, Dec:green) between the coadd and the different individual exposures
  %where each source has been detected.}
  %\label{fig:astrometry_internal}
%\end{figure}

%\subsection{Photometry checks}
%\label{sec:photometry_checks}

%As we did in the previous subsection, we perform two different tests to assess the quality of our simulations: first, we compare our output catalogs to the inputs, and second, we check the consistency between different visits for the same objects. \rachel{I'm not sure we can fully frame this as a test of the simulation quality.  It's also a test of the DM processing.  Errors in either stage could affect the photometry and in fact it can be difficult to identify the origin of errors that are found.}

%\subsubsection{External checks}
%\label{sec:external_photometry}

%We want to analyze the accuracy of the magnitude measurement by comparing the input catalog with the output catalog. This process not only checks
%the accuracy of the measurement pipeline, but it also tests the quality and level of realism of the image generation pipeline.
%For these external checks, we use the same 50 randomly selected visits from the previous section.
%To study the photometric residuals, we use a different matching strategy compared to previous sections. In this case, we eliminate the threshold
%in magnitude difference, looking for the input source that is closest in magnitude within a 0.2 arcseconds radius around each detected
%source. \rachel{Source of any kind?  Or star?  The figure 8 label suggests it's stars, but the text here is ambiguous.} In \figref{photometry_a} we can see the distribution of the photometric residuals. This distribution gets wider as we go
%fainter (as expected). Furthermore, we see that the 0.02 magnitude selection cut is a good proxy to ensure that we account for most sources properly matched. \rachel{Only down to 20th magnitude, not beyond.}

%\begin{figure}
 % \centering
  %\includegraphics[width=0.45\textwidth]{photometry_imsim_dithered_50visits_hist}
  %\includegraphics[width=0.45\textwidth]{photometry_imsim_dithered_50visits}
  %\caption{Top: Distribution of magnitude difference between the input and output catalogs.
  %Bottom: Difference magnitude between input and output catalogs as a function of the measured magnitude. We considered 50 visits
  %randomly selected from the imSim dithered run. We find similar results for the imSim undithered run and for the PhoSim run. \rachel{I think this caption may be old - we aren't going to talk about the PhoSim results here, are we?}}
  %\label{fig:photometry_a}
%\end{figure}

%\subsubsection{Internal checks}
%\label{sec:internal_photometry}

%We also checked the consistency of the photometry between different exposures. Using the same approach and sample presented in
%\secref{internal_astrometry}, we compared the mean difference between the coadded source and 10 random single-visit images. In
%principle, we would expect that some objects will show differences between different epochs due to their intrinsic variability, however,
%these effects are not found in the inputs of the simulation for any of the objects, thus, the differences between different epochs are due to
%statistical fluctuations and different observing conditions. The results from these checks are shown in \figref{internal_photometry_a}.

%\begin{figure}
 % \centering
  %\includegraphics[width=0.45\textwidth]{photometry_internal_10visits_imsim_undithered}
  %\caption{Distribution of the mean difference in magnitude between the coadd and the different individual exposures
  %where each source has been detected. The mean of the distribution is consistent with zero.}
  %\label{fig:internal_photometry_a}
%\end{figure}

%In this figure, we see that the mean of the distribution is compatible with zero and that most sources have a magnitude difference lower
%than 25 mmag given the narrow peak. The presence of long tails is likely due to blends and mismatching or artifacts.

%\subsection{PSF checks}
%\label{sec:psf_checks}

%In order to ensure the accuracy of shape measurements, a robust and accurate estimation of the PSF is required. In the case of imSim, we have
%an analytical input PSF that only depends on the airmass at the time of observation. We can check if the measurement pipeline can reconstruct
%the input PSF. To do so, we select 200 visits randomly, retrieve the measured PSF using the pipeline, and compare this
%to the input model given the observing conditions. We obtain the residual depicted in \figref{psf_residual}, where we see that the PSF model residuals are very small in the center of the image and that the tails are dominated by noise. \rachel{Are you going to use any of the standard PSF model diagnostics, like fractional size residuals, shape residual distributions, $\rho$ statistics?  It would be useful to have a quantitative rather than qualitative statement here.}

%\begin{figure}
%\centering
%\begin{tikzpicture}
 % \node[inner sep=0pt] (A) {\includegraphics[width=0.9\columnwidth]{psf_residual.png}};
 % \node[black] (B) at ($(A.south)!-.01!(A.north)$) {X [pixels]};
 % \node[black,rotate=90] (C) at ($(A.west)!-.01!(A.east)$) {Y [pixels]};
 % \node[black,rotate=90] at ({$(A.west)!1.0!(A.east)$} |- {$(A.south)!.55!(A.north)$}) {PSF$_{output}-$PSF$_{input}$};
%\end{tikzpicture}

%\caption{Average difference between the input PSF model (41$\times$41 pixels, normalized to unity) and the measured PSF (with the same normalization) for 200 visits.}
%\label{fig:psf_residual}
%\end{figure}

\subsection{LSST Science Requirements Document Key Performance Metrics}

After doing some basic sanity checks, we processed the output individual-visit catalogs through the LSST Project package \texttt{validate\_drp}\footnote{\url{dmtn-0008.lsst.io}, \url{https://github.com/lsst/validate_drp}}.
The \texttt{validate\_drp} package calculates the Key Performance Metrics (KPMs) from the LSST Science Requirements Document\footnote{\url{https://ls.st/LPM-17}}~\citep{LPM-17}, which we will refer to as LSST-SRD. This document contents science-driven requirements for LSST data products and we use it as a guide to check the status of our end-to-end pipeline. In addition, we will also validate our dataset using some of the requirements at the DESC Science Requirement Document~\citep[][DESC-SRD;]{2018arXiv180901669T}, which we will refer to as DESC-SRD.
 
By design, our simulations do not satisfy some of the requirements in the LSST-SRD, such as the number of filters in the surveyed fields (Filter complement in Table~\ref{tab:kpm_table}) or the number of filters (Nfilt) used in a given night because we only simulate images in $r$-band. On the other hand, our images automatically meet some criteria due to the design choices. For example, the maximum pixel size (pixSize) allowed in the LSST-SRD is 0.22 arcseconds and our simulations have a fixed pixel size of 0.2 arcseconds. Finally, some requirements in both the LSST-SRD and the DESC-SRD cannot be tested with these single-band images, we will ignore such tests in this work.

We start by checking the absolute astrometry by comparing the input and output catalogs. We find a mean (and median) deviation of 38 milliarcseconds between the input and measured positions. This is due to the fact that the corrections for proper motion were not present in the version \texttt{13.0} of the DM stack. However, this bias is still below the design specifications for the absolute astrometric performance of LSST as defined by the LSST-SRD, denoted by AA1, which is 100 milliarcseconds. This can be seen in~\figref{AA1}.
\textcolor{red}{Check with Chris and others: Why does the proper motion bias astrometry in one direction??}

\begin{figure}
\centering
\includegraphics[width=0.9\columnwidth]{astrometric_residuals_single_visit_2d}
\caption{Astrometric residuals when comparing with the input catalogs for 500 randomly selected visits in the dithered simulation. We find a bias of 38 milliarcseconds in the measured right ascension due to uncorrected proper motion. We obtain similar results using the undithered simulation.}
\label{fig:AA1}
\end{figure}

We also validate the photometric repeatability by comparing the measured magnitude of bright unresolved objects across different visits, i.e., we select stars with signal-to-noise ratio (SNR) $> 100$. The photometric repeatability (PA1) is 6~mmag as measured by the interquartile range (IQR $\equiv$ 75th percentile minus 25th percentile) and is shown in \figref{validate_drp_PA1}. The minimum specification is 8~mmag so, our dataset and pipeline behave as required. The 6~mmag is dominated by a tail of scattered objects, see \figref{validate_drp_check_photometry}.

The photometric error model from \citet[][Eq. 4,5]{Overview} is 
\begin{eqnarray}
\sigma^2 = \sigma^2_{\rm sys} + \sigma^2_{\rm rand} \\
\sigma^2_{\rm rand} = (0.04 - \gamma) &10^{0.4(m-m_5)} + \gamma 10^{0.8(m-m_5)}
\label{eq:photometric_error}
\end{eqnarray}
where $\gamma$ describes noise from the sky and detector electronics, and the 5-$\sigma$ point-source depth is given by:
\begin{equation}
\begin{split}
m_5 = &C_m + 0.50 (m_{\rm sky} - 21~{\rm mag/{\rm arcsec}^2}) \\
&+ 2.5 \log_{10} (0.7{\arcsec}/\theta_{\rm eff}) \\
&+ 1.25 \log_{10} (t_{\rm vis} / 30~{\rm s}) - k_m (X-1)
\label{eq:photometric_m5}
\end{split}
\end{equation}
where $C_m$ summarizes the throughput of the telescope optics and camera, $m_{\rm sky}$ is the sky brightness, $\theta_{\rm eff}$ is the seeing, $t_{\rm vis}$ is the exposure time, $k_m$ is the airmass coefficient, and $X$ is the airmass.

Our fits find $m_5$ = 24.2~mag/arcsec$^2$ and $\gamma=0.038$ which are quite consistent with the \citet[][Table 2]{Overview} values of $m_5=24.35$~mag/arcsec$^2$, $\gamma=0.039$.  This demonstrates that we generate simulations of the telescope, detector, and sky in accordance with the \citet{Overview} estimates.

Note that we find $\sigma_{\rm sys}=0$~mmag, which is consistent with the idealized simulations we run for DC1 with no additional systematic sources of errors. The LSST-SRD also includes criteria about the percentage of bright, unresolved sources (PF1$= 20\%$) with a photometric measurement deviation larger than a certain threshold (PA2 = 15 mmag). We check this by computing the fraction sources in~\figref{validate_drp_PA1} that deviate more than 15 mmag, obtaining PF1$=17\%$ in compliance with the requirements.

\begin{figure}
\centering
\includegraphics[width=0.9\columnwidth]{DC1-imsim-dithered_r_PA1.png}
\caption{(left) The magnitude difference of pairs of measurements of stars across visits for stars with a typical SNR $>100$.  (right) The histogram of these differences.  The Gaussian root mean square (RMS) is shown in red while the interquartile range is shown in green. Note that the distribution is more peaked than a Gaussian. The interquartile range (6.2~mmag) is {\em smaller} than the Gaussian RMS (15.8~mmag).}
\label{fig:validate_drp_PA1}
\end{figure}

\begin{figure*}
\centering
\includegraphics[width=1.8\columnwidth]{DC1-imsim-dithered_r_check_photometry.png}
\caption{Photometric performance in more detail.  (upper left) Horizontal histogram of the RMS distribution for each star for the full sample (grey) and ``bright'' sample (blue).  The blue sample is the same as that shown in Figure~\ref{fig:validate_drp_PA1}. (upper right).  Note that this RMS is the RMS of the measurements for a given star and so each entry in the histogram is one star. Thus, the median value here is 2.4~mmag, even thought the median of all RMS measurements in Figure~\ref{fig:validate_drp_PA1} was 15.8~mmag.  This difference is due to a population of objects that have a poor repeatability. (upper right) The RMS distribution as a function of magnitude of the star.  Here you can see the tight core of very repeatable measurements, along with a smaller population of objects with much higher scatter to their measurements. (lower left) Per-object median reported magnitude error by the pipeline versus the observed RMS over all of the measurements for an object.  Our $SNR>100$ cut is in the space of median reported error. (lower right) Per-object median pipeline-reported magnitude error versus object magnitude. 
{\bf TODO: MWV CLEANUP (1) title, (2) incorrect xlabel (should be \#/bin),  and "Filter name" extra text}
{\bf TODO: MWV  Are these higher scatter objects galaxies, stars in halos of bright stars, additionally added objects that were done inconsistently with the rest of the catalog?}
{\bf TODO: MWV  The lower-left plot has never actually made sense to me.  I should investigate this more.}
}
\label{fig:validate_drp_check_photometry}
\end{figure*}

We continue by checking that the zeropoints, $z_{p}$ fulfill the uniformity requirements specified in the LSST-SRD. More concretely, they should have show an RMS lower than 15 mmag (PA3) and no more than 20\% (PF2) of the images should have a deviation larger than 20 mmag (PA4). We randomly select 1000 sensor-visits and check the distribution of the zeropoints, depicted in~\figref{PA34}, finding that the RMS of the distribution is 37 mmag, not fulfilling the requirement. This is not worrisome since we are not interested in the colors of the sources given that we only have single band images. We also find that 19\% of the sensors have a zeropoint that deviates from the median more than 20 mmag in compliance with the requirements. Finally, we compare input and output magnitudes and compute the median difference between both (PA6), finding it to be 17 mmag and smaller than the maximum allowed in the LSST-SRD (20 mmag). More details about how the matching between inputs and outputs was done in order to perform this test can be found in Section~\ref{sec:matching}.

\begin{figure}
\centering
\includegraphics[width=0.9\columnwidth]{PA234.pdf}
\caption{Normalized distribution of the change in the zeropoint value, $\Delta z_{p}$ for 1000 randomly selected sensor-visits. We see that the distribution is quite asymmetrical and find that the RMS (PA3) is 37 mmag, larger than the requirements in the LSST-SRD (15 mmag). We also find that only 19\% (PF2) of the visits deviate from the median more than 20 mmag (PA4), in compliance with the LSST-SRD requirements.}
\label{fig:PA34}
\end{figure}

We now focus on checking the astrometric performance.  We show the detailed astrometric repeatability in \figref{validate_drp_check_astrometry}, where detected stars that are detected at different visits are matched and their astrometric residual is shown as a function of their SNR. The LSST-SRD sets astrometric requirements for stellar pairs at a separation $D$ in different exposures. The RMS of the separation between these pairs should not exceed AMx milliarcseconds and less than AFx \% of the sample should deviate more than ADx milliarcseconds from the median. AMx, AFx, and ADx are specified for pairs separated by D=5, 20 and 200 arcmin for x=1,2 and 3. The results of these checks can be found in~\figref{validate_drp_AMx}.

\begin{figure}
\centering
\includegraphics[width=0.9\columnwidth]{DC1-imsim-dithered_r_check_astrometry.png}
\caption{
Detailed astrometric performance.  (left) Histogram of the repeatability of distance between pairs of stars.  (right) Astrometric variation as a function of SNR.  We find a systematic floor of 12~milliarcsec.
{\bf TODO: MWV CLEANUP (1) title, (2) fix xlabel, which should be \#/bin}
{\bf TODO: MWV IS THIS HISTOGRAM DIFFERENT THAN THE check\_photometry histogram in terms of per-pair vs. per-object variation.}}
\label{fig:validate_drp_check_astrometry}
\end{figure}

\begin{figure}
\centering
\includegraphics[width=0.9\columnwidth]{{DC1-imsim-dithered_r_validate_drp.AM1_D_5_arcmin_17.0_21.5_mag}.png}
\includegraphics[width=0.9\columnwidth]{{DC1-imsim-dithered_r_validate_drp.AM2_D_20_arcmin_17.0_21.5_mag}.png}
\includegraphics[width=0.9\columnwidth]{{DC1-imsim-dithered_r_validate_drp.AM3_D_200_arcmin_17.0_21.5_mag}.png}
\caption{
Astrometric repeatibility: variation in distances measurement between pairs of stars at (top) 4-6$\arcmin$, (middle) 19-21$\arcmin$, (bottom) 199-201$\arcmin$.  AM(1,2,3) are the distance measurements, while AF(1,2,3) are the fraction of pairs lying outside the a specified limit AD(1,2,3).
The performance is excellent, with characteristic values all below the LSST-SRD levels.
{\bf TODO: MWV CLEANUP}}
\label{fig:validate_drp_AMx}
\end{figure}

In addition, the LSST-SRD sets some requirements in the minimum per-visit image depth for images with fiducial sky brightness of 21 mag/arcsec$^2$, exposure time of 30 s, airmass=1 and fiducial seeing (FWHM) of 0.7 arcseconds. In order to mimic this we select the visits that fulfill the following criteria:
\begin{itemize}
\item Altitude $> 80$ degrees.
\item $0.68\arcsec <$ seeing (FWHM) $ < 0.71\arcsec$
\item Sky-brightness (in $r$-band) $ \geq 21$ 
\end{itemize}LSST-SRD
We obtain a total of 520 sensor-visits fulfilling these criteria. We then compute the median 5-$\sigma$ depth following the methodology described later in this work and compare with the predicted depth by \texttt{OpSim}. After this, we check that the median of the depth distribution is larger the minimum depth (D1=DB1=24.3 mag) as defined in the LSST-SRD. We also check that no more than 20\% of the visits (DF1) have a depth lower than 24.0 (Z1) by computing the 20th percentile. The results of these checks are depicted in \figref{DF1_checks}. We find that the median of the depth distribution in the selected visits, $24.297 \pm 0.009$ mag, is compatible (within a standard deviation of the mean) with the minimum depth set in the LSST-SRD. We find as well that the 20th percentile 24.1 is larger than the minimum value (Z1) set in the LSST-SRD. We also check that in a given visit, the variation in the field-of-view is within the requirements. The LSST-SRD establishes that, in a representative visit with depth D1 no more than (DF2) 20\% of the field-of-view will be (Z2) 0.4 magnitudes brighter than the nominal (24.3). We select visit 2218486 since its median depth is 24.3. We find that the 20-th percentile is 24.29 fulfilling the criteria.

\begin{figure}
\centering
\includegraphics[width=0.9\columnwidth]{m5_goals}
\caption{Measured depth in visits with Altitude $>80$ deg., $0.68 \arcsec <$ seeing $ < 0.71 \arcsec$ and Sky-brightness (in $r$-band) $\geq 21$ (blue histogram) compared to the predicted depth by \texttt{OpSim} (solid orange histogram). The median of this distribution (dashed line) is very close to the LSST-SRD minimum depth D1=24.3 (red vertical line), the 20th percentile is also shown and we can appreciate that is larger than Z1=24.0 as established by the LSST-SRD.}
\label{fig:DF1_checks}
\end{figure}

The LSST-SRD also sets criteria regarding the maximum modulus of the PSF ellipticity, $|e|$,
for exposures with PSF-FWHM $\approx 0.69\arcsec$ and no more than 10 degrees apart from zenith, where $|e|$ follows the distortion definition~\citep{1991ApJ...380....1M}:
\begin{equation}
|e| = \frac{a^{2} - b^{2}}{a^{2}+b^{2}}
\end{equation}
were $a, b$ are the semi-major and semi-minor axes of the 
in particular, it requires that the median is no larger than 0.05 (SE1) and that no more than 10\% (EF1) of the images exceed 0.1 (SE2). Our analytic (and circularly-symmetric) PSF models should, by design, fulfill these criteria. However, we have to test if the reconstructed PSF also fulfills them. The PSF was reconstructed using the PSFEx~\citep{2011ASPC..442..435B} implementation in the LSST software stack. We tested this in the processed data by using the same 520 sensor-visits used to check the DB1 and DF1 criteria. We checked the modulus of the PSF ellipticity in the position of the detected objects in these visits accumulating them in the histogram shown in \figref{SE1_DC1}. We obtained that SE1=0.001 and SE2=0.002, way below the maximum values allowed by the LSST-SRD. In future data challenges, we plan to increase the complexity and realism of the PSF to be closer to the maximum values set by these requirements.

\begin{figure}
\centering
\includegraphics[width=0.9\columnwidth]{PSF_ellipticity_DC1}
\caption{PSF ellipticity distribution accumulated for 520 sensor-visits measured at the position of detected objects. The median (0.001) and 90th (0.002) percentiles are shown as the dashed lines. Note that these values are an order of magnitude lower than the specifications by the LSST-SRD (0.05 and 0.1, respectively).}
\label{fig:SE1_DC1}
\end{figure}

We also checked that, in these images, 85\% of the point-like sources flux is contained within $0.80\arcsec$ or less (SR1), 95\% within or less $1.31\arcsec$ (SR2) and 99\% within $1.81\arcsec$ (SR3) or less. In our case we obtain SR1$=0.64\arcsec$, SR2$=1.01\arcsec$ and SR3$=1.79\arcsec$. This was done by calculating the radius at which the PSF was at its 85th, 95th and 99th percentiles.

For weak lensing analyses, the correct modeling of the PSF is crucial~\citep{2004MNRAS.353..529H} and both the LSST-SRD and the DESC-SRD contain explicit requirements about the PSF residuals. In particular, the LSST-SRD states that using the full survey data the auto- and cross-correlations ($E_{1}, E_{2}, E_{X}$) of the PSF residuals over an arbitrary field-of-view should be below TE1 (3 $\times 10^{-5}$) for $\theta \leq 1$ arcmin, below TE2 ($3 \times 10^{-7}$) for $\theta \geq 5$ arcmin and no more than TEF (15) \% of the images will have median larger than TE3 ($6 \times 10^{-5}$) for $\theta \leq 1$ arcmin and, TE4 ($5 \times 10^{-7}$) for $\theta \geq 5$ arcmin. 

To check these criteria we calculate $E_{1}, E_{2}, E_{X}$ in using the definitions of the LSST-SRD:
\begin{eqnarray}
e_{1} = \frac{\sigma^{2}_{1} - \sigma^{2}_{2}}{\sigma_{1}^{2}+\sigma_{2}^{2}},\\
e_{2} = \frac{2\sigma^{2}_{12}}{\sigma_{1}^{2}+\sigma_{2}^{2}},\\
E_{1} (\theta) = \langle \delta e^{(i)}_{1}\delta e^{(j)}_{1} \rangle,\\
E_{2} (\theta) = \langle \delta e^{(i)}_{2}\delta e^{(j)}_{2} \rangle,\\
E_{X} (\theta) = \langle \delta e^{(i)}_{1}\delta e^{(j)}_{2} \rangle,
\end{eqnarray}
With $\sigma_{1}^{2}, \sigma_{2}^{2}$ are the second order moments of a source along some set of perpendicular axes and $\sigma^{2}_{12}$ is the covariance, $\delta e_{1}, \delta e_{2}$ are the residuals, and the angle brackets indicate averaging over all pairs of stars $i$, $j$ at a given angular separation $\theta$.

In practice, we compute the PSF-corrected moments of stars across the field of view using \texttt{TreeCorr}~\citep{2004MNRAS.352..338J}. Our findings are shown in \figref{TEx}. We can see that for $E_{1}$ and $E_{X}$ the TE1 criteria is fulfilled but the TE2 and others are not. For $E_{2}$ there is a residual flat correlation. \textcolor{red}{Check with Robert/Rachel: This is likely a result of the our analytic PSF input model having some residual shape that cannot be described by the Gaussian profiles that the DM stack used, more suitable for real data?}
\begin{figure}
\centering
\includegraphics[width=0.9\columnwidth]{TEx}
\caption{Auto and cross-correlation functions, $E_{1}$ (blue), $E_{2}$ (orange), $E_{X}$ (green) of the PSF residuals as a function of the aperture angle $\theta$. We see that our data does not fulfill the LSST-SRD requirements.}
\label{fig:TEx}
\end{figure}

Finally, the DESC-SRD requires (WL4-Y10) that the systematic uncertainty in the PSF model defined using the trace of the second moment matrix should not exceed 0.1\% for full-depth (Y10) DESC weak lensing analysis. We randomly select 3000 visits, obtain the input and measured PSF, and measure the trace of the second order moments, $T$ with \texttt{GalSim}. We then compute the relative difference, $\Delta T/T$, obtaining the results depicted in \figref{WL4-Y10}. We find that our dataset shows that the standard deviation of the distribution is 0.05\% lower than the requirement.
\begin{figure}
\centering
\includegraphics[width=0.9\columnwidth]{WL4-Y10}
\caption{Normalized distribution of the relative difference in the trace of the second order moments, $\Delta T/T$ between input and output PSF. We see that the standard deviation of the distribution (WL4-Y10) is 0.05\%, much complying with the requirement (0.1\%).}
\label{fig:WL4-Y10}
\end{figure}

We summarize our findings in Table~\ref{tab:kpm_table} and Table~\ref{tab:desc_srd_table}. We see that our images and catalogs pass most of the requirements by a good margin. We can also see that some of the criteria are not passing the requirements due to design choices and do not affect the science analyses that can be performed with this dataset. However, there is a big exception: the TEx requirements (correlations of the residual PSF ellipticity) and WL4-Y10 prevent possible weak lensing analyses. In any case, given that no cosmic shear signal was included in these simulations, the simplicity of the potential sources of systematic uncertainty for cosmic shear analyses, and the lack of photometric redshifts for the sources, this dataset does not offer any new science from the WL point-of-view and no WL analyses are performed on this work. 

\begin{table}[h!]
\centering
\begin{tabular}{|c|c|c|c|}
\hline
Quantity (LSST-SRD) & Minimum & DC1 & Passed\\
\hline
Filter complement & ugrizy & r & \textcolor{blue}{N}\\
Nfilters & 3 & 1 & \textcolor{blue}{N}\\
Nv1 & 184 & 184 & \textcolor{blue}{Y}\\
pixSize (arcsec) & 0.22 & 0.2 & \textcolor{blue}{Y}\\
AA1 (milliarcsec) & 100 & 20 & Y\\
PA1 (millimag) & 8 & 6 & Y\\
PF1 (\%) & 20 & 17 & Y\\
PA2 (millimag) & 15 & - & -\\
PA3 (millimag) & 15 & 37  & N\\
PF2 (\%) & 20 & 19 & Y\\
PA4 (millimag) & 20 & - & -\\
%PA5 (millimag) & 10 & 1.4 & check\\
PA6 (millimag) & 20 & 17 & Y\\
AM1 (milliarcsec) & 20 & 8 & Y\\
AF1 (\%) & 20 & 13 & Y\\
AD1 (milliarcsec) & 40 & - & -\\
AM2 (milliarcsec) & 20 & 4 & Y\\
AF2 (\%) & 20 & 9 & Y\\
AD2 (milliarcsec) & 40 & - & -\\
AM3 (milliarcsec) & 30 & 7 & Y\\
AF3 (\%) & 20 & 2 & Y\\
AD3 (milliarcsec) & 50 & - & -\\
%Fleak (\%) & 0.02 & 0 & \textcolor{blue}{Y}\\
%FleakTot (\%) & 0.1 & 0 & \textcolor{blue}{Y}\\
D1 (mag) & 24.3 & 24.3 & Y\\
DF1 (\%) & 20 & - & -\\
Z1 (mag) & 24.0 & 24.1 & Y\\
DB1 (mag/r-band) & 24.3 & 24.3 & Y\\
DF2 (\%) & 20 & - & -\\
Z2 (mag) & 0.4 & 0.1 & Y\\
%S1 (0.44) & 0.59 & check & check\\
%S1 (0.60) & 0.72 & check & check\\
%S1 (0.80) & 0.89 & check & check\\
%SF1 (\%) & 10 & check & check\\
%SX & 1.2 & check & check\\
%SXE & 0.59 & check & check\\
SE1 & 0.04 & 0.001 & Y\\
EF1 & 10 & - & -\\
SE2 & 0.1 & 0.002 & Y\\
SR1 (arcsec) & 0.80 & 0.64 & Y\\
SR2 (arcsec) & 1.31 & 1.01 & Y\\
SR3 (arcsec) & 1.81 & 1.79 & Y\\
TE1 & $3 \times 10^{-5}$ & $1.2\times 10^{-4}$ & N\\
TE2 & $3 \times 10^{-7}$ & $1.3\times 10^{-4}$ & N\\
TEF & 15\% & - & -\\
TE3 & $6 \times 10^{-5}$ & - & N\\
TE4 & $5 \times 10^{-7}$ & - & N\\
\hline
\end{tabular}
\caption{Table with the Key Performance Metrics (KPMs) that we used to check the quality of our end-to-end pipeline. The criteria that are passed or not due to design choices are shown in blue. We mark with dashes the quantities that we use as auxiliary to compute some of the criteria, e.g., we use EF1 to compute SE2.}
\label{tab:kpm_table}
\end{table}

\begin{table}[h!]
\centering
\begin{tabular}{|c|c|c|c|}
\hline
Quantity (DESC-SRD) & Maximum & DC1 & Passed \\
\hline
WL4-Y10 (\%) & 0.1 & 0.05 & Y\\
%WL5-Y10 (\%) & 0.1 & N/A & N/A\\
\hline
\end{tabular}
\caption{Criteria from the DESC-SRD tested in the DC1 simulation. This document also includes other criteria that cannot be tested with our data but will be relevant for future data challenges.}
\label{tab:desc_srd_table}
\end{table}

%\rachel{Is this all the KPMs?  I thought some have to do with PSF model quality, etc.}
%\rachel{Note that there are many figures associated with this subsection, but the text doesn't actually refer to or discuss any of them.}

\section{Data selection and masking}
\label{sec:data_selection}
In this section, we describe how we take advantage of the fact that we have full knowledge about the simulated sources in order to get a ``clean" data sample, we also describe the catalog mask and the maps generated for the different observational effects present in the simulation. Unless explicitly stated, the procedures and selections made in this section will be performed in both the dithered and undithered simulations.

\rachel{One persistent source of confusion in this section (and to some extent the previous one) is the question of what sample is being used.  This confusion shows up in two ways: first, a failure to state what selection criteria were imposed (flags, bright star masks, etc.) or magnitude cuts or \dots so as to avoid bad detections (which affects the interpretation).  Second, a lack of clarity - there are places where it says ``objects'' or ``sources'' and then later in the same section it is implied that a sample of galaxies or stars were used.  I would recommend that somewhere before you start showing results, you have a subsection that clearly defines what samples there are, how exactly they are defined, give them unique names, and use them consistently throughout the text.  Finally, I would note that in some of the places where you mention what selection criteria are used, some flags that I normally think of as important to avoid junk detections are not mentioned, so I would encourage a reconsideration of the flags (and a careful interpretation in light of whatever flag cuts we do use in the end).}

\subsection{Matching inputs and outputs}
\label{sec:matching}

Using end-to-end simulations, one can potentially trace each measured photon to its corresponding source and fully characterize the image generation and measurement processes. In practice, this is very difficult to do because of the large data overhead (either each photon would have to be tagged somehow in the final image or each pixel would have to contain information about the sources that contributed to it). 

The simplest way to connect two catalogs is by using the positions of the objects in the sky. This approach has been extensively used in the literature~\citep{1977A&AS...28..211D,1983Obs...103..150B,1986MNRAS.223..279W} and performs reasonably well when blending is low (i.e, when the amount of overlapping sources in the image is low). However, when blending is high, this approach might be not be sufficient. Then, matching other quantities like flux or shape can become useful~\citep{2008ApJ...679..301B}.

We compare two different matching strategies: positional matching, where we find the objects in the truth catalog closest to the detected objects, which we will refer to as \textit{pure spatial matching} and will denote as \textsf{S}; and positional matching with magnitude matching, which we will refer to as \textit{spatial+magnitude matching} and denote as \textsf{S+M}, where for each detected object we find objects from the input catalog that lie within a three pixel radius ($0.6 \arcsec$). After this, we select the object that is closest in magnitude as long as the difference in magnitude is less than a certain threshold. In our case, we conservatively choose $1.0$. Using this approach, if none of the neighbors fulfill these conditions, the detected source is considered unmatched. Note that in the case of pure spatial matching all sources are matched to an object in the true catalog.

In both cases, we build a \texttt{KDTree}~\citep{scikit-learn} using the positions of detected objects with \texttt{detect\_isPrimary=True} which ensures that the source has been fully deblended and was detected in the inner part of the coadd regions (see~\citet{2018PASJ...70S...5B} and ~\citet{2018PASJ...70S..25M} for more details). In order to speed up the processing and to reduce the usage of computational resources we build the \texttt{KDTree} using output sources from 30 randomly selected coadd regions (patches) in the dithered simulation\footnote{The undithered simulation yields the same results and conclusions.} containing 975,605 detected sources fulfilling the aforementioned condition (we will refer to these as primary outputs or primary detected sources). Using this sample, we find that 95\% of these sources are matched using the spatial+magnitude matching (by construction all of them are matched using the pure spatial matching approach). An interesting metric is to check the fraction of primary detected sources that have been matched to the same exact object in the input catalog, which we will denote as $f_{\mathrm{multi}}$. This is an unusual occurrence, however, imagine the case where an spurious fluctuation has been marked as a source, this fluctuation is primary detected source but has no counterpart in the input catalog, another example are bright objects that have been shredded and are detected as several fainter sources. We find that these kind of matches is 100 times more likely to happen using the pure spatial matching ($f_{\mathrm{multi}} = 3 \times 10^{-3}$) than in the spatial+magnitude matching ($ f_{\mathrm{multi}} = 2 \times 10^{-5}$). This is expected because, in the cases where the primary detected source is a random fluctuation or a shredded source, it is unlikely that the measured flux is close to the flux of a neighboring source thus producing an unmatched source in the case of using spatial+magnitude matching. We also find that both approaches select the same best matching source in the input catalog for only 68\% of the primary detected sources for which we found a suitable match. These differences can be due to several reasons, e.g. objects with a poorly determined centroid positions that are close to other objects in the input catalog (remember that the input catalog contains objects up to $r=28$), objects with poorly determined fluxes (low SNR), etc. 

We compare as well the astrometric and photometric residuals using both approaches in~\figref{matching_comparison}. We can see that the median astrometric residuals are similar in both cases and that the search radius (3 pixels=600 mas) for the spatial+magnitude matching is large enough to map the overall distribution. On the other hand, the resulting median photometric residual seems strongly biased in the case of using just pure spatial matching, and the scatter is very large as shown by the error-bars. However, in the case of spatial+magnitude matching, the median residuals and their error-bars are smaller, as expected given the maximum magnitude threshold. We can see that in this case the biases are still significant. This can be mitigated using a smaller tolerance in magnitude (for example using 0.5 mag instead of 1 mag) but that implies an overall reduction of the number of matched primary detected sources. Finally, taking a look at the magnitude difference between inputs and outputs of individual matched sources using the spatial+magnitude technique, we can see that this distribution is centered around zero, except for very bright sources ($r < 17$), where saturation prevents us from having an accurate determination of the fluxes. 

\begin{figure}
\centering
\includegraphics[width=0.9\columnwidth]{astrometry_residual_comparison}
\includegraphics[width=0.9\columnwidth]{photometry_residual_comparison}
\caption{Top: Normalized distribution of measured astrometric residuals (in the RA axis but similar results are found for the Dec. axis) for pure spatial matching (blue) and spatial+magnitude matching (orange), the median values of each distribution are marked by dashed lines and dashed-dotted lines respectively. Bottom: Median per-bin photometric residual as a function of measured magnitude for pure spatial matching (green), and spatial+magnitude matching (orange) using 30 magnitude bins between $r$-band magnitude 10 and 30. We also show the individual residuals for the matched objects using spatial+magnitude matching (blue dots).}
\label{fig:matching_comparison}
\end{figure}

At the end of the day, different matching techniques have different potential applications and strengths. In our case, we are going to use these matching techniques to provide a clean (flux-limited) sample to perform two-point clustering analyses. Given that magnitude precision and accuracy will be important for our sample selection, we will use the spatial+magnitude matching technique to clean the sample from artifacts and poorly measured sources.

\subsection{Sample selection}
\label{ssec:sample_selection}

In this subsection we are going to use the spatial+magnitude technique to identify flags or thresholds in variables that may allow us to get a clean sample for clustering. In principle, given that the LSST DM software stack is essentially the same as the reduction pipeline used in~\citet{2018PASJ...70S..25M}, we could potentially use similar cuts. However, note that we are working in $r$-band only so some of the required cuts cannot be performed. In addition to this, some variables, like the so-called blendedness parameter, are not available in the version of the LSST DM software stack that we used to process the data. As a consequence, we propose our own selection cuts, although we use the criteria in~\citet{2018PASJ...70S..25M} as guidance.

The methodology to perform the selections is simple: we check the primary detected sources that have no match using the spatial+magnitude technique and we compute the fraction of objects that are flagged, $f_{u,i} = N_{flag_{i}, unmatched}/N_{total, unmatched}$, and compare it to the corresponding fraction of flagged matched primary detected sources, $f_{m,i} = N_{flag_{i}, matched}/N_{total, matched}$, for each one of the flags, $flag_{i}$, in the catalog. If the ratio $f_{u,i}/f_{m,i}$ is larger than 50 for a particular flag and $f_{m,i} < 0.01$, i.e., less than 1\% of the matched primary sources have that flag, it means that the presence of that flag is a good indicator of problematic sources and get rid of the sources with those flags. We also repeat the same procedure looking for the absence of a certain flag or whether a quantity is frequently measured as \texttt{NaN} and check for redundant cuts.

As a result, we are going to eliminate from our selection all the sources that fulfill one of the following conditions:
\begin{itemize}
\item \texttt{detect\_isPrimary = False}. As discussed earlier, this means that the source has not been fully deblended or is outside of the inner region in a coadd.
\item \texttt{base\_NaiveCentroid\_flag = True}. This means that there is a general failure during the source measurement.
\item \texttt{base\_SdssShape\_flag\_psf = True}. This means that there is a failure in measuring the PSF model shape in that position.
\item \texttt{ext\_shapeHSM\_HsmSourceMoments\_flag\_not\_contained = True}. This means that the center of the source is not contained in its footprint bounding box.
\item \texttt{modelfit\_DoubleShapeletPsfApprox\_flag = True}. This means that there is a general failure while performing the double-shapelet approximation to the PSF model in the position of this source (see Appendix 2 in~\citet{2018PASJ...70S...5B} for more details).
\item \texttt{base\_PixelFlags\_flag\_interpolated = True}. This means that there are interpolated pixels in the source's footprint.
\item \texttt{base\_PixelFlags\_flag\_interpolatedCenter = True}. This means that the center of a source is interpolated.
\item \texttt{base\_PixelFlags\_flag\_saturatedCenter = True}. This means that the center of a source is saturated.
\item \texttt{base\_ClassificationExtendedness\_flag = True}. This means that there is a general failure when using the extendedness classifier.
\item \texttt{modelfit\_CModel\_flags\_region\_usedInitialEllipseMin = True}. This means that the pixel region for the final model fit is set to the minimum bound used in the initial fit.
\item \texttt{base\_SdssShape\_x/y = NaN}. This means that the centroid position (either in the x or y axes) is measured as NaN.
\item \texttt{base\_SdssCentroid\_x/yErr = NaN}. This means that the error in the centroid position (either in the x or y axis) is measured as NaN.
\end{itemize}

After these cuts we keep 8.25 million objects in the dithered catalog and 7.51 million objects in the undithered catalog. We will refer to this sample as the \textit{clean sample}. 

We now focus on how many of these objects are matched as a function of magnitude and signal-to-noise ratio. In \figref{snr_mag_selection}, we can see that the fraction of unmatched objects for $r < 26$ is very low and that after this, it greatly increases. We also see that for objects with SNR$>6$ the fraction of unmatched objects starts to be negligible. These two thresholds ($r < 26$ and/or SNR$>6$) look like natural selection cuts to ensure good quality data. However, we will analyze further these magnitude and SNR thresholds before imposing any cuts to the sample in subsequent sections.

\begin{figure}
\centering
\includegraphics[width=0.9\columnwidth]{unmatched_fraction_magnitude.pdf}
\includegraphics[width=0.9\columnwidth]{unmatched_fraction_SNR.pdf}
\caption{Top: Ratio of unmatched to matched primary detected sources after selection cuts as a function of magnitude. The dashed vertical lines show the median depth for the dithered (orange) and undithered (green) fields. Bottom: Ratio of unmatched to matched primary detected sources after selection cuts as a function of SNR.}
\label{fig:snr_mag_selection}
\end{figure}

\subsection{Depth maps and footprint masking}
\label{sec:masking}

In order to estimate the depth in the coadd catalogs we generate a HEALPix\footnote{\url{http://healpix.sf.net}}~\citep{2005ApJ...622..759G} map containing the detected primary sources in the \textit{clean sample} from the previous subsection. Then, for each HEALPix pixel, we compute the median SNR of the objects in that pixel as a function of the magnitude and use the magnitude at which SNR is closest to 5.

These maps are shown in \figref{depth_maps}. We can see that the dithered simulation is indeed very uniform ($> 50\%$ of its footprint lie in the same exact depth bin) showing the success of the dither strategy. However, we can see that in the undithered simulation, as expected, there are zones with a higher depth (the overlap of the pointing positions) and a reduced median depth.
\begin{figure}
\centering
\includegraphics[width=0.9\columnwidth]{dithered_depth.png}
%\includegraphics[width=0.45\textwidth]{dithered_difference.png}
\includegraphics[width=0.9\columnwidth]{undithered_depth.png}
%\includegraphics[width=0.45\textwidth]{undithered_difference.png}
\includegraphics[width=0.9\columnwidth]{depth_comparison_1d.pdf}
\caption{5-$\sigma$ depth for the dithered (top) and undithered (middle) fields. There is an increased depth in the overlapping parts of the pointings in the undithered field but the median depth is lower. We also show the 1D distribution of depth (bottom) for both fields for easier comparison.}
\label{fig:depth_maps}
\end{figure}

We also check the depth by checking the detection efficiency of stars as a function of magnitude. To do so, we use the stars in the input catalog and select those that lie within the simulated footprint, after this, we match them using the spatial+magnitude technique described in previous sections. Finally, we compute the number of detected objects in the \textit{clean sample} matched to stars divided by the number of stars in the input catalog as a function of the magnitude of the matched object in the input catalog. The results can be seen in \figref{stellar_detection_efficiency}. We see that there is a high efficiency $> 80\%$ up to $r \approx 26$. 
\begin{figure}
\centering
\includegraphics[width=0.9\columnwidth]{stellar_detection_efficiency.pdf}
\caption{Ratio of number of detected objects matched to stars in the \textit{clean sample} and number of stars in the input catalog as a function of magnitude for the dithered (blue) and undithered (orange).}
\label{fig:stellar_detection_efficiency}
\end{figure}

Given the results in the previous subsection and this section, we decide to use only the galaxies that lie in pixels with limiting magnitude $r \geq 26$ and select those objects with magnitude $r \leq 26$. This cut is roughly equivalent to selecting $SNR > 6$ and thus, we eliminate most of the artifacts in the sample. After these cuts we obtain 7.16 and 6.17 million objects for the dithered and undithered fields respectively.

\subsection{Bright object masking}

Bright objects produce significant effects in the image that affect the detection and measurement of neighboring objects. Some examples of these effects include saturation, large diffraction spikes, obscuration of neighboring sources, etc. Thus, masking a region around these sources creates a more complicated footprint but greatly simplifies the analysis of systematic effects. 

In order to evaluate the effects, we follow an analogous process to those described in \citet{2018PASJ...70S...7C} and \citet{2018PASJ...70S..25M}. Using the position of stars in the input catalog, that lie within the considered footprint, and with input magnitudes in the range $m_{1} < r < m_{2}$ we count all primary detected objects (avoiding any other selection cuts) in a given radius $\theta$, and compute the median of the set, $N_{neighbors}$. We repeat this for different radii and magnitude ranges. Finally, we repeat this process for all stars in the input catalog in the footprint and compute $N_{neighbors, tot}$ and compute the ratio $y = N_{neighbors}/N_{neighbors, tot}$. We fit this ratio to an error function:
\begin{equation}
y_{model} = \mathrm{Erf}\left({\theta/\theta_{mask}}\right)
\end{equation}
This fit is performed for the first three bins in stellar magnitude that we consider, $r < 14$, $14 < r < 16$, and $16 < r < 18$. We ignore the points with $\theta < 10$ arcsec for the fit. Following \citet{2018PASJ...70S...7C} we use the point where the density reaches 95\% as masking radius, i.e., $r_{mask,fit} = \mathrm{Erf}^{-1}(0.95) \theta_{mask}$. After this, we compare to equation (1) in \citet{2018PASJ...70S..25M}:
\begin{equation}
r_{mask,HSC} = 200\times 10^{0.25(7-m_{BS})} + 12 \times 10^{0.05(16-m_{BS})} 
\end{equation}
Where $m_{BS}$ is the magnitude of the bright star to mask. Note that this prescription is specific for HSC. However, the LSST instrument and observing conditions are different. This is why we decided to rescale $r_{mask,HSC}$ by the ratio of mean seeing in LSST and HSC.
\begin{equation}
r_{mask,LSST} = r_{mask,HSC} \frac{1.04\arcsec}{0.58\arcsec}
\end{equation}
Where 1.04\arcsec is the mean seeing in DC1, and 0.58\arcsec is the mean seeing reported in~\citet{2018PASJ...70S..25M}. We obtain the results depicted in \figref{bright_object_masking}. We see that the error function describes well the behavior for $N_{neighbors}/N_{neighbors,tot}$. We also see that the masking radius, $r_{mask,LSST}$, using the rescaled prescription set by \citet{2018PASJ...70S..25M} is a pretty good approximation for our data. In particular, we find that the ratio $r_{mask,fit}/r_{mask,LSST} \approx 1.14$. This 14\% disagreement can be related to the fact that we chose a very simple rescaling based on the mean seeing rather than the delivered PSF size. As a result, we decide to use $r_{mask, DC1} = 1.14 r_{mask, LSST}$ around stars in the input catalog with $r < 16$, since this is the saturation limit for LSST. We generate a high-resolution mask using \texttt{HEALPix} with $N_{side}=16384$, i.e. $\approx 12.9 \arcsec$, then we downsample this map to $N_{side}=2048$ and get rid of the objects that lie within pixels that have more than 50\% of their area masked. This results into a total area loss of $\approx 27.5\%$ of our footprint, which can be seen in \figref{bo_mask}. After these cuts we keep 5.27 million in the dithered field and 4.52 million in the undithered field.
\begin{figure}
\centering
\includegraphics[width=0.9\columnwidth]{bright_object_masking}
\caption{Ratio of the median number of primary detected objects neighboring a star in a certain magnitude range in the input catalog to the median number of objects detected any star in the input catalog, as a function of the distance to the star $\theta$. Different symbols and colors represent different magnitude ranges for the stars in the input catalog considered. The shadowed regions represent the best fit masking radius, $r_{mask,fit}$ with their uncertainties and the vertical lines correspond to $r_{mask,LSST}$ evaluated at the mean magnitude in each bin.}
\label{fig:bright_object_masking}
\end{figure}
\begin{figure}
\centering
\includegraphics[width=0.9\columnwidth]{bo_mask}
\caption{Bright object mask. This map shows with 1 (yellow) the pixels that we will consider for our analysis, and with 0 (blue) the regions of the sky that will not be considered in our analysis using this mask. We use a high-resolution ($\approx 12.9 \arcsec$) map to mask around bright stars ($r < 16$) and then, we downsample the map to a lower resolution ($\approx 1.7 \arcmin$) and get rid of the pixels where the masked area by bright stars is higher than 50\% of the pixel.}
\label{fig:bo_mask}
\end{figure} 
\subsection{Blending}
As previously mentioned, our output catalogs do not include any estimation of overlap between sources, or \textit{blendedness}~\citep{2018PASJ...70S...5B}. Highly-blended objects are more likely to have biased estimations of the centroid position, shape, and fluxes. This can lead to overall biases in the estimated photometric redshifts and cosmological parameters. Given that the mean seeing in DC1 is larger than in HSC, the impact of blended objects is likely to be larger. In the case of \citet{2018PASJ...70S..25M}, the cut in the blendedness parameter affects only 1\% of the objects; we expect this number to be larger in our case. Using specialized image simulations from~\citet{Sanchez19} with the LSST nominal seeing in $r$-band, we find that if we select objects with $r < 26$ and SNR $>1$, the fraction of objects with blendedness $ < 10^{-0.375}$ is $\approx 6.3\%$. If we raise the minimum SNR to 6, this fraction is lowered to $\approx 2.4\%$. We do not expect that the inclusion of these objects in our two point measurements will affect the range of scales that we are going to consider in this work. In order to quantify this, a careful study of the impact of blending in clustering measurements is necessary but this is out of the scope of this work.

\section{Two-point clustering results}
\label{sec:results}

\rachel{Big picture question: This subsection starts out ``In this section, we analyze the two point clustering statistics\dots'' - but you don't.  This is a subsection.  It's only in the next subsection that you define and show plots of the two-point statistics.  So I think either this subsection should be combined with the next one, {\em or} the two subsections should be promoted to a section together, perhaps with 6.6 as well.  Perhaps combining the subsections would be better, keeping them within this ``Analysis'' section.}

In this section, we analyze the two point clustering statistics for both the dithered and undithered catalogs in real and harmonic space and check the consistency between the input and measured observables. When comparing the input and output catalogs, some subtleties arise. For example, selecting a galaxy sample in the input catalog with $m_{\rm low} < m_{\rm true} < m_{\rm high}$, where $m_{\rm low}$ is the lower $r$-band magnitude threshold, $m_{\rm true}$ is the true $r$-band magnitude of the object and $m_{\rm high}$ is the upper $r$-band magnitude threshold, and a second galaxy sample in the output catalog with $m_{\rm low} < m_{\rm measured} < m_{\rm high}$, with $m_{\rm measured}$ being the $r$-band measured magnitude, results in a different power-spectra. The power spectrum in the first (selected according to true magnitude) sample will have a larger overall power than the second (selected according to measured magnitude) sample. This is due to the fact that the second sample includes galaxies that are fainter, resulting in a wider window function. \rachel{This sentence is pretty mystifying.  You haven't said how/why does the output catalog are fainter, given the same magnitude cuts.  Presumably, you are referring to the impact of noise, which will preferentially scatter in more faint galaxies than it will cause galaxies to be lost on the bright end.  But what about blends, can we reason about what that should do?  Also, the connection between ``include fainter galaxies'' and ``wider window function'' isn't really made.  I presume you are connecting fainter galaxies with higher redshift, but again, it's better to be explicit.}

Using the samples presented in previous sections \rachel{You've defined several different samples in previous sections, so this statement is not entirely clear.}, we measure the 2-point correlation function using \texttt{TreeCorr}~\citep{2004MNRAS.352..338J}, selecting the Landy \& Szalay estimator~\citep{1993ApJ...412...64L},
\begin{equation}
w(\theta) = \frac{DD - 2 DR + RR}{RR}
\end{equation}
where $DD, DR$, and $RR$ are the number of pairs of objects from the data $D$ or the random catalog $R$ that cover the footprint. \rachel{How was the random catalog constructed?  What physical effects does it include - masking, depth variations, \dots?} We use 30 angular log-spaced bins between $\theta=0.0001^{\circ}$ \rachel{Maybe say what this is in arcsec/arcmin since that will be more natural to most readers?  Also, why so small?} and $\theta=10^{\circ}$ \rachel{My gut reaction is that for a 40 square degree survey it doesn't make sense to go to 10 degrees.  You should have basically no pairs there.  The plot only goes to 1 degree (which still feels somewhat aspirational) so I was wondering if this  ``10'' is a typo and you really just went to 1?}. We choose the number of bins so that the resulting covariance matrix is nearly diagonal. The covariance matrices are calculated using the delete-one jackknife technique~\citep{Shao:1986:DJB,2009MNRAS.396...19N}. We divide the footprint in $N_{JK}=100$ regions. These regions are defined using the K-means algorithm from the package \texttt{kmeans\_radec}\footnote{\url{https://github.com/esheldon/kmeans\_radec}}. The covariance matrix is computed as
\begin{equation}
\mathrm{Cov}_{JK}(\theta_{i},\theta_{j})=\frac{N_{JK}-1}{N_{JK}}\sum_{k=1}^{N_{JK}}\Delta w_{k}(\theta_{i}) \Delta w_{k}(\theta_{j})
\end{equation}
\begin{equation}
\Delta w_{k}(\theta_{i}) = w_{k}(\theta_{i})-\bar{w}(\theta_{i})
\end{equation}
where $w_{k}(\theta_{i})$ is the value of the correlation function when deleting the $k$-th region at the scale $\theta_{i}$, and $\bar{w}(\theta_{i})$ is the average correlation function at that same scale. We compute the correlation function on the dithered and undithered catalogs using their respective footprints, and we compare with the predicted correlation function given the $N(z)$ obtained by the spatial matching presented in previous sections, using \texttt{CCL}\footnote{\url{https://github.com/LSSTDESC/CCL}}\citep{2018arXiv181205995C}. \rachel{The theory predictions also require a galaxy bias, so what galaxy bias vs.\ redshift was used in these theory predictions?} The results are shown at \figref{2pt_corr}. We find that both measurements are in good agreement and in agreement with the theoretical prediction for $\theta < 0.3$ degrees. The angular range is mainly limited by our surveyed area. \rachel{How to explain the residual large-scale correlations?}

\begin{figure}
\centering
\includegraphics[width=0.9\columnwidth]{w_comp_corr25p3.pdf}
\includegraphics[width=0.9\columnwidth]{sys_dithered_25p3_v2.pdf}
\includegraphics[width=0.9\columnwidth]{sys_undithered_25p3_v2.pdf}
\caption{{\bf Top panel:} Results for the two-point correlation function in the input (green), dithered (blue) and undithered (orange) datasets. We can see that after the correction (triangles) for the systematic effects, presented in Section \ref{ssec:systematics}, the agreement is better than without the correction (circles) between the input and output data especially for the undithered case (for the dithered case the corrections are negligible). {\bf Middle and bottom panels:} Correction due to the different potential sources of systematic uncertainty relative to the value of the measured correlation function of the dithered (middle panel) and undithered (bottom panel) datasets. We show that the correction is at percent level for $\theta<0.03$ degrees and grows beyond the $20\%$ level for $\theta \approx 0.1$ degrees. \rachel{It should say which simulation was used directly on the panels as well.} \rachel{How should we interpret the fact that the correction is incredibly noisy on larger scales?  Does that mean it's not even possible to derive  a correction there?  50\% of the plot is in the region with noisy corrections, so it is worth reconsidering what scales are worth showing.}}
\label{fig:2pt_corr}
\end{figure}

%\begin{figure}
%\centering
%\includegraphics[width=0.9\columnwidth]{w_dithered_25p3.pdf}
%\includegraphics[width=0.9\columnwidth]{w_undithered_25p3.pdf}

%\caption{Correction due to the different potential sources of systematic uncertainty relative to the value of the measured correlation function of the dithered (top) and undithered (bottom) datasets. We show that the correction is at percent level for $\theta<0.03$ degrees and grows beyond the $20\%$ level for $\theta \approx 0.1$ degrees.}
%\label{fig:sys_realspace}
%\end{figure}


%We also show the correlation matrix,
%\begin{equation}
%\mathrm{Corr}_{ij}=\frac{\mathrm{Cov}_{ij}}{\sqrt{\mathrm{Cov}_{ii}\mathrm{Cov}_{jj}}}
%\end{equation}
%in \figref{2pt_cov} where we can see that in both cases the matrices are almost diagonal.
%\begin{figure}
%\centering
%\includegraphics[width=0.9\columnwidth]{correlation_matrix_dithered_25p3_v2.pdf}
%\includegraphics[width=0.9\columnwidth]{correlation_matrix_undithered_25p3_v2.pdf}
%\caption{Correlation matrices for the dithered (top) and undithered catalogs (bottom).}
%\label{fig:2pt_cov}
%\end{figure}
\subsubsection{Angular power spectrum}
We measure the angular power spectrum using the \texttt{NaMaster}\footnote{\url{https://github.com/LSSTDESC/NaMaster}} package~\citep{2018arXiv180909603A}. \texttt{NaMaster} computes the cross-power-spectra of spin-0 and spin-2 maps, with an arbitrary mask and number of contaminants using a pseudo-$C_{\ell}$ approach~\citep{2002ApJ...567....2H,2017MNRAS.465.1847E}. In this case, we calculate the power spectrum in the range $0 < \ell < 6144$ and $\Delta \ell = 75$. The choice of $\Delta \ell$ was done so the resulting covariance matrix is close to diagonal. We calculate the covariance matrices with three different approaches: The first approach is the same delete-one jackknife technique performed in real space; in the second approach, we compute the covariances using the mode-counting formula~\citep{Dodelson:1282338,2007MNRAS.381.1347C}
\begin{equation}
\mathrm{Cov}_{\ell\ell'}=\frac{2}{f_{\rm sky}\Delta\ell}\left(\frac{C_{\ell}^{2}}{2\ell+1}+\frac{1}{\bar{n}^{2}}\right)\delta_{\ell\ell'}
\end{equation}
where $\bar{n}$ is the number density (objects per steradian). For the third approach we compute the Gaussian covariance with \texttt{NaMaster}. These three methods give consistent results. We compute the theoretical prediction for the power-spectra with \texttt{CCL}:
\begin{equation}
C_{\ell}^{\rm TH} = \frac{2}{\pi}\int{dz} \left(\frac{dn(z)}{dz}\right)^{2} b^{2}(z) \int{dk k^{2} P(k,z)j^{2}_{\ell}(kr(z))}
\end{equation}
where $P(k,z)$ is the power spectrum, $b(z)$ is the bias and $\frac{dn}{dz}$ is the number density as a function of redshift. We use the Millenium cosmological parameters~\citep{2005Nature.435.629S} ($\Omega_{m}=0.25$,$\Omega_{b}=0.045$,$\Omega_{\Lambda}=0.75$, $n=1$, $\sigma_{8}=0.9$, $h=0.73$), and the $dn/dz$ built with the input catalog using the same magnitude cuts. \rachel{Need to say what you used for galaxy bias.} The results are depicted in \figref{power_spectra}. We see that the results for both datasets follow the theoretical prediction within errors. \rachel{Interpretation of this result, given the limitations such as mismatch between input/output samples, galaxy bias assumptions, \dots?}
\begin{figure}
\centering
\includegraphics[width=0.9\columnwidth]{Cl_25p3_errors}
\includegraphics[width=0.9\columnwidth]{Cl_25p3_sys_comparison}
\caption{{\bf Top panel:} Measured power spectra undithered (open orange circles) and dithered (solid blue circles) datasets with \texttt{NaMaster} corrected by systematics. The error bars are computed using the Gaussian approximation. All datasets seem to be compatible with the theoretical prediction (green line) within 1-$\sigma$. {\bf Bottom panel:} Correction in harmonic space due to the different potential sources of systematic uncertainty relative to the value of the measured correlation function of the dithered (solid blue) and undithered (dashed orange) datasets}
\label{fig:power_spectra}
\end{figure}
\subsubsection{Systematic effects}
\label{ssec:systematics}
In this section, we analyze  different potential systematic errors affecting the DC1 data. We consider the following observational quantities as sources for systematic uncertainties:
\begin{itemize}
\item Extinction: The CatSim catalog provides the value for the magnitudes corrected for extinction using the map from \citet{1998ApJ...500..525S}, which we refer to as the SFD map.
\item Stellar contamination: In this case, we build a HEALPix map using the input CatSim stellar catalog.
\item Sky-background: We use the observed background level in each exposure and assign that value to the HEALPixel with $N_{\rm side}=2048$ that corresponds to the center of the field of view for each visit. After this we calculate the median value in each HEALPixel to build a map. The caveat of this approach is that we are not propagating the geometry of the focal plane.
\item Sky-noise: We use the observed noise background level in each exposure and proceed as in the previous case to build a HEALPix map with which we cross-correlate.
\item Seeing: We proceed as before and use the observed seeing in each exposure and build a HEALPix map.
\end{itemize}
These maps are shown in \figref{systematic_maps} and \figref{systematic_maps2}. \rachel{We've jumped from figure 21 to 25, so perhaps some reordering is needed?}

In the case of the real space measurements, we proceed as in \citet{2016MNRAS.455.4301C} to compute the impact of the different potential sources for systematic uncertainty. We compute the auto and cross-correlations of these maps and our data samples to obtain the corrected correlation function:
\begin{equation}
w^{gg}(\theta)_{\rm corr} =w^{gg}(\theta)_{\rm obs} -  \vec{w}_{g,{\rm sys}} \cdot W_{\rm sys,sys}^{-1} \cdot \vec{w}_{g,{\rm sys}}
\end{equation}
where $w^{gg}_{\rm corr}$ is the corrected galaxy-galaxy correlation function, $w^{gg}_{\rm obs}$ is the measured galaxy-galaxy autocorrelation, $\vec{w}_{g,{\rm sys}}$ is the vector containing the cross-correlation between the galaxies and the different maps, and $W_{\rm sys,sys}^{-1}$ is the inverse of the matrix containing the cross-correlations between different systematics. For the stellar contamination, we also follow the procedure presented in~\citet{2016MNRAS.455.4301C}. Given a stellar fraction $f_{\rm star}$, the ``true'' galaxy correlation function is given by
\begin{equation}
w_{\rm gal} = \left(1+f_{\rm star}\right)^{2}\left(w_{\rm obs} - f_{\rm star}^{2}w_{sg} - \frac{f_{\rm star}^{4}}{\left(1+f_{\rm star}\right)^{2}}\right)
\end{equation}
where $w_{sg}$ is the cross-correlation between our stellar map and the observed galaxies. This assumes that the stellar contamination is proportional to the fraction of stars misclassified as galaxies (with a factor that can be dependent on the observing conditions) and that the true correlation between stars and galaxies is zero.

We select those objects classified as galaxies whose centroids lie within 2 pixels of a star from the input catalog and a measured magnitude within 30 mmags of that from the input star. By doing this, we estimate a stellar contamination of $f_{\rm star}=7.2\%$ in our sample. \rachel{I don't understand this at all.  Can you explain?  You're saying that 7.2\% of the galaxy sample is actually stars?  This is totally inconsistent with work from HSC with our fiducial selection criteria, so I suspect that our cuts used in this analysis are not good.  This result is so bad compared to what can be done using the LSST stack that I think anybody would say we should be redefining our cuts, not just using this sample as-is.}

We compare the correction term, $\vec{w}_{g,{\rm sys}} \cdot W_{\rm sys,sys}^{-1} \cdot \vec{w}_{g,{\rm sys}}$, for the different systematics to the measured signal obtaining the results in the middle and bottom panels of \figref{2pt_corr}. We show that the dither strategy works to reduce the impact of systematic effects at the percent level for $\theta < 0.03$ degrees in this ``small'' area, improving the signal-to-noise. \rachel{I do not fully understand the statement about signal-to-noise, can you explain?} This is also seen at the top panel of \figref{2pt_corr}, where we clearly see that the uncorrected and corrected correlation function for the dithered field are essentially the same.

%\begin{figure}
%\centering
%\includegraphics[width=0.9\columnwidth]{Cl_25p3_sys_comparison}
%\caption{Correction in harmonic space due to the different potential sources of systematic uncertainty relative to the value of the measured correlation function of the dithered (solid blue) and undithered (dashed orange) datasets.}
%\label{fig:sys_harmonic_space}
%\end{figure}


In the case of harmonic space, we use the mode deprojection from \texttt{NaMaster}~\citep{2018arXiv180909603A} to correct for the different templates. In the bottom panel of \figref{power_spectra}, we show the relative size of the correction due to the presence of the considered sources of systematic uncertainty. Again, we see that the dithered strategy diminishes the effect of the potential sources of systematic uncertainty in the power-spectrum measurements.

%\CHECK{Add figure with corrected and uncorrected power-spectra.}

%\subsection{Pushing the dataset further}

%We also wanted to check what happens if we choose a sample where the uniformity across the footprint is still good even though completeness is not as good. We extended our sample by selecting galaxies with \texttt{ $16 < $ CMODEL\_MAG $ < 26.25$} and depth$\geq 26.25$. This increases the sample from 4.3 million galaxies to 6.5 million in the dithered sample, even though the footprint is reduced by $8\%$. However, these selection criteria reduce the number of galaxies to 4.2 million in the undithered sample, as well as the footprint by $35\%$.

%\subsection{Reconstructing the selection function}
%As mentioned before, one potential application of this kind of end-to-end study is the possibility of analyzing how the pipeline performs data detection and measurement. In terms of the two-point statistics, we could think of this as a selection (window) function. In this section we are going to try to reconstruct the selection function given our input and output catalogs.
%
%We use the spatial matching in \secref{matching} to get the magnitude and the error distribution for our detected galaxies. Then, we bin our sample in 100 true-magnitude bins from 18 to 28, and model the difference between input and output flux, $\Delta F$=$F_{\rm meas}$-$F_{\rm true}$, in each bin using a Gaussian mixture model (GMM) with 6 components for objects brighter than 26.4, and 3 components for fainter objects. We chose this threshold since it is close to our faintest $5\sigma$ limiting magnitude. This way, we have 100 models (one per magnitude bin) to distort the true magnitude and get a reconstructed (or emulated) magnitude. An example of this GMM is shown in \figref{example_GMM}. We see that, in a given bin, the GMM captures the overall distribution of $\Delta F$ well, allowing us to model the flux errors as a function of magnitude.
%\begin{figure}
%\centering
%\includegraphics[width=0.9\columnwidth]{example_GMM}
%\caption{Measured $\Delta F$ (blue solid circles) compared to 69,000 samples of the Gaussian mixture model that we obtain in the bin $22.6 <$ mag$_{\rm true} < 22.7$. We repeated this for 100 bins between $18 <$ mag$_{\rm true} < 28$.}
%\label{fig:example_GMM}
%\end{figure}
%
%\rachel{The rest of this subsection is one extremely long paragraph.  Need to break it up into paragraphs with well-defined chunks of info.}
%There are several caveats with this approach: $\Delta F$ is not Gaussian in the center of the distribution, which means that the GMM, regardless of its number of components, will show a slightly larger error than with real data. The second, bigger, caveat is that the models are constructed from detected objects but no information from the non-detected objects biasing the modeled $\Delta F$ is included. When we try to impose some cuts in our reconstructed magnitudes, we recover a larger number of objects than the detected number. This is shown in \figref{emulated_magnitudes}, where we use the emulated magnitudes to resemble the selection made in previous sections, i.e., including selected objects with $21.4 \leq r_{\rm emulated} \leq 25.3$. We see that up to $r _{\rm true} \approx 26$, we have a high level of completeness. We also see that our emulation, even though it gives us a larger number of detected objects, does a good job following the shape of the magnitude distribution. Above $r_{\rm true}=26$, we are dominated by the outliers in the distribution of $\Delta F$ and we detect more objects than we should. We can also see that, if we only use objects such that $21.4 \leq r_{\rm true} \leq 25.3$, we obtain a biased estimation of the two-point statistics, given that in this case we ignore the (heavy) tails of the magnitude distribution.
%\begin{figure}
%\centering
%\includegraphics[width=0.9\columnwidth]{emulated_magnitude_histogram}
%\caption{True magnitude distribution for the detected objects using spatial matching (blue) and for the emulated objects using the same cuts in measured and emulated magnitude, $25.3 \geq r \geq 21.4$.}
%\label{fig:emulated_magnitudes}
%\end{figure}
%In \figref{emulated_power} we show that the power-spectrum using the true magnitude cuts is very close to the power-spectrum of the detected objects. However, at small scales, it is smaller than the power-spectrum of the detected objects. This is likely due to the presence of fainter objects in the detected  sample. In addition, blended objects can create a small contribution at these scales. Finally, the larger shot-noise can make the residuals at small scales larger. On the other hand, at large scales, the power-spectrum using the true magnitude cuts is slightly larger due to the smaller magnitude range (and likely redshift range) considered. We also see that the power-spectrum for the emulated magnitude cuts is noticeably lower than the one for the detected objects. This is due to the larger magnitude range considered in the emulated sample (so we are averaging over a bigger volume in general). It is clear that this procedure is not good to emulate the selection function of our pipeline, given the sensitivity of LSST. We also see that performing cuts on the true magnitude distribution is not enough to recover the measured power-spectrum at the level of precision that we will have with LSST data. Therefore, more sophisticated ways of emulating the processing pipeline should be implemented, and tools like \texttt{BALROG}~\citep{2016MNRAS.457..786S}, \texttt{imSim}, or \texttt{PhoSim}~\citep{2015ApJS..218...14P} that generate synthetic images applied to random (zero-power-spectra) catalogs are interesting for future studies, in an effort to achieve percent-level sensitivities.
%\begin{figure}
%\centering
%\includegraphics[width=0.9\columnwidth]{emulated_power_spectra}
%\caption{Shot-noise subtracted measured power-spectra for the detected objects by the LSST stack (solid blue circles), the objects in the true catalog spatially matched (open orange circles), the objects with $21.4 \leq r_{\rm true} \leq 25.3$ (red crosses), and the objects with $21.4 \leq r_{\rm emulated} \leq 25.3$ (green triangles). \rachel{Legend shows magnitude ranges in other direction - should flip them around in the legend.}}
%\label{fig:emulated_power}
%\end{figure}
% ----------------------------------------------------------------------

\section{Conclusions}
\label{sec:conclusions}

\rachel{Perhaps emphasize further some tests of the pipeline performance?}

End-to-end simulations are very useful tools to test the overall performance of any current and future cosmological experiments like the LSST~\citep{Overview}. They allow us to test and improve different parts involved in the data processing and analysis, as well as, to model and improve our control of systematic uncertainties.

In this paper, we have presented a simulated (end-to-end) imaging dataset that resembles single-band, full-depth LSST data, for the first data challenge in the LSST DESC. We simulated images using state of the art tools (\textit{imSim}). We generated two different and complementary datasets, one with random dithers (\textit{dithered}) and the other with no dithers (\textit{undithered}). We processed these images with the LSST DM stack. We performed several quality assurance tests on the outputs from the DM stack, including photometry and astrometry checks. We checked that both the \textit{dithered} and \textit{undithered} are high-quality datasets with good photometry and astrometry (with the caveat of uncorrected proper motion).

We studied different ways to relate the output catalogs to the inputs. In particular, we studied two different matching strategies. The first, using information about positions only; and a second strategy involving positions and magnitudes. We showed that the angular power-spectrum is recovered at a very high-level of precision in both cases. However, these matching techniques are likely not good enough for studies about blending or very small scale information since they do not include information about undetected sources present in blends. An important research topic for these kind of end-to-end simulations is to find efficient strategies to relate inputs and outputs.

Finally, we estimated the depth of our simulated catalogs and selected a high-completeness sample to perform clustering analysis in both real and harmonic space. The results of this analysis, indicate that the simulated foregrounds have a low impact at the scales considered for our study, i.e.\ lower than $5\%$ for $\theta < 0.03^{\circ}$ (or for $100 < \ell < 6000$), especially in the \textit{dithered} dataset. This indicates the success of the dither strategy considered in this study. We also conducted a pilot study to try to reconstruct the selection function of our pipeline. We showed that simple methods cannot capture the complexity of our data given the high sensitivity that we will have in LSST and concluded that more complex methods should be studied in order to create better models.

The methodology presented in this work will serve as basis for future DESC data challenges, where we aim to perform multi-band studies in a larger area, analyze complementary image generation strategies (\texttt{PhoSim}), and increase the complexity of the foregrounds included.

% ----------------------------------------------------------------------

\subsection*{Acknowledgments}

This research used resources of the National Energy Research Scientific Computing Center, a DOE Office of Science User Facility supported by the Office of Science of the U.S. Department of Energy under Contract No. DE-AC02-05CH11231. We acknowledge the use of \texttt{Pandas, Dask, SciPy, Matplotlib, Jupyter, CCL, NaMaster, Healpy, and scikit-learn} as well as the LSST software stack.

The work of JC, RD, SD, TG, AJ, HK, PJM and BVK was supported by the U.S. Department of Energy under contract number DE-AC02-76SF00515. 
The work of RM was supported by the US Department of Energy Cosmic Frontier program, grant DE-SC0010118.

% 
This is the text imported from \code{acknowledgments.tex}, and will be replaced by some standard
LSST DESC boilerplate at some point. \rachel{Should use https://github.com/LSSTDESC/desc-tex/blob/master/ack/standard.tex}
% 


%{\it Facilities:} \facility{LSST}

% Include both collaboration papers and external citations:
\bibliography{lsstdesc,main}

\appendix
\section{Mapping observational effects}
\label{sec:systematic_maps}
\begin{figure}
\centering
\includegraphics[width=0.45\columnwidth]{median_seeing.png}
\includegraphics[width=0.45\columnwidth]{median_skybg.png}
\includegraphics[width=0.45\columnwidth]{median_skynoise.png}
\includegraphics[width=0.45\columnwidth]{extinction.png}
\includegraphics[width=0.45\columnwidth]{stellar_density.png}
\caption{HEALPix maps showing the different observational effects that might be potential cause of systematic uncertanties.}
\label{fig:systematic_maps}
\end{figure}
%\begin{figure}
%\centering

%\caption{HEALPix maps showing the different observational effects that might be potential cause of systematic uncertanties.}
%\label{fig:systematic_maps2}
%\end{figure}
%\section{Comparison of depth estimation methods}
%\begin{figure}
%\centering
%\includegraphics[width=0.9\columnwidth]{dithered_difference.png}
%\includegraphics[width=0.9\columnwidth]{undithered_difference.png}
%\caption{Relative difference between the depth calculated using the two methods presented in the text for the dithered (top) and undithered (bottom) fields.}
%\label{fig:depth_comparison}
%\end{figure}
\end{document}


% ======================================================================
%
